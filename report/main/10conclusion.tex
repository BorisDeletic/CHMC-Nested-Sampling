%! Author = borisdeletic
%! Date = 14/05/2023

% Preamble
\documentclass[11pt]{article}

% Document
\begin{document}

    \section{Conclusion}\label{sec:conclusion}
    We have now introduced our new algorithm and implementation for Constrained Hamiltonian Monte Carlo for nested sampling.
    By drawing concepts from classical mechanics together with Bayesian inference we provide a powerful
    set of tools which are very effectively applied to lattice field theory.

    A summary of our novel algorithms and results include
    \begin{itemize}
        \item Constrained Hamiltonian Monte Carlo
        \begin{itemize}
            \item Epsilon halving: a new algorithm for guaranteeing a valid reflection in discretized solvers.
            \item Reflection rate hopping: a new mechanism for moving dead points from local to global maxima.
            \item Epsilon adaption: a dual averaging scheme for continuously adapting the step size throughout evolution.
            \item Metric adaption: A new method to adapt the kinetic energy metric to maintain efficient sampling at every iteration.
        \end{itemize}
        \item First application of nested sampling to $\phi^4$ theory to calculate a numerically verified phase diagram.
        \item Calculation of magnetisation and correlation functions for $128 \times 128$ sized lattices with nested sampling.
        \item Demonstrated computationally viable nested sampling for likelihoods up to dimensions $D=500,000$.
    \end{itemize}


    Further work can be done in parallelising the C++ implementation to take full advantage of modern compute clusters.
    Using auto differentiation will also require further investigation to ensure no pathological behaviour arises for
    complex likelihoods.

    By exploiting gradients we believe it is possible to use this algorithm for a new class of
    Bayesian problems previously inaccessible to machines.

\section*{Acknowledgements}
    I would like to acknowledge my supervisors, who have provided invaluable guidance and support throughout my
    foray into physics research.

    I would also like to thank Dr Will Handley for his helpful insights
    and for staying to answer my questions after his lectures.

    Finally, I would like to thank my housemates, who all also study physics.
    They have proved invaluable friends who are always there to make you laugh while debugging code together at midnight.
\end{document}