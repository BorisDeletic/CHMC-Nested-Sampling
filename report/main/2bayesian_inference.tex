%! Author = borisdeletic
%! Date = 08/05/2023

% Preamble
\documentclass[11pt]{article}

% Document
\begin{document}

    \section{Bayesian Inference}\label{sec:bayesian_inference}
    Bayesian inference is a robust analytic framework which allows the construction of predictive models $\mathcal{M}$
    in the context of some dataset $\mathcal{D}$.

    The \emph{likelihood} is defined as the probability of observing the data given a specific parameter choice $\theta$
    \begin{equation}\label{eq:likelihood}
        p(\mathcal{D} | \theta, \mathcal{M}) \equiv \mathcal{L}(\theta).
    \end{equation}
    A Bayesian model must also specify it's distribution of parameters before any data is known.
    This is termed the \emph{prior}, defined by
    \begin{equation}\label{eq:prior}
        p(\theta|\mathcal{M}) \equiv \pi(\theta).
    \end{equation}
    The \emph{evidence} is the distribution of observed data marginalised over the parameters, defined as
    \begin{equation}\label{eq:evidence}
        p(\mathcal{D} | \mathcal{M}) \equiv \mathcal{Z} = \int{\mathcal{L}(\theta) \pi(\theta) d\theta}.
    \end{equation}
    The evidence, or sometimes \emph{marginalised likelihood}, is an important quantity which provides a measure
    on the quality of a model $\mathcal{M}$.

    By using Bayes' Theorem~\cite{bishop2006}, the distribution of parameters $\theta$ given our model and data
    can be written in terms of the quantities
    \begin{equation}\label{eq:bayes_theorem}
    \begin{aligned}
        p(\theta | \mathcal{D}, \mathcal{M}) &=
            \frac{p(\mathcal{D} | \theta, \mathcal{M}) p(\theta|\mathcal{M})}{p(\mathcal{D} | \mathcal{M})}, \\
        \mathcal{P}(\theta) &= \frac{\mathcal{L}(\theta) \pi(\theta)}{\mathcal{Z}},
    \end{aligned}
    \end{equation}
    where $\mathcal{P}(\theta)$ is termed the \emph{posterior}.
    The posterior is the distribution of parameters $\theta$ after taking the data into account.

    An intuitive relationship between these quantities is that a model where the \emph{prior} more closely
    resembles the \emph{posterior} will have a greater \emph{evidence}~\cite{mackay2003}.

    Calculating the posterior is in the domain of \emph{parameter estimation}, which is often a very difficult task
    to solve analytically.
    For high-dimensional problems, we therefore resort to computationally estimating the distribution by sampling
    from the posterior using Markov-Chain Monte-Carlo techniques~\cite{gupta2014comparison, delmoral2013mean}.

    Examples of such sampling algorithms include Metropolis-Hastings~\cite{Metropolis_OG},
    Slice sampling~\cite{neal2003slice}, and Hamiltonian Monte Carlo~\cite{HMC_Duane, neal1996monte} which
    we explore further in section~\ref{sec:hamiltonian_monte_carlo}.


\end{document}