%! Author = borisdeletic
%! Date = 06/05/2023

% Preamble
\documentclass[11pt]{article}

% Document
\begin{document}

    \appendix
    \section{Constrained HMC Parameter Values}\label{sec:param_table}
    The below table shows the values used for all input parameters into CHMC as defined in the paper.
    $\delta = 0.8$~\cite{MCMChamiltonian}

    \section{Metric Adaption with the Equipartition Theorem}\label{sec:metric_derivation}
    Consider a $D$ dimensional Hamiltonian with kinetic and potential energy
    \begin{equation}\label{eq:hamiltonian_appendix}
        H = \frac{1}{2} \mathbf{p}^T M^{-1} \mathbf{p} + U,
    \end{equation}
    with momentum distributed as $\mathbf{p} \sim \mathcal{N}(0, \sigma=M)$.
    For simplicity, we use a scaled unit metric $M = \alpha \mathbb{1}$.
    According to the equipartition theorem, the energy is equally shared among the degrees of freedom.
    Therefore, we match the variance in potential energy per dimension to the kinetic energy.
    \begin{equation}\label{eq:var_matching}
    \begin{aligned}
        \frac{1}{D} \mathrm{Var}[U] &= \mathrm{Var}[T]  \\
        \mathrm{Var}[U] &= D\mathrm{Var}\left[\frac{1}{2} \mathbf{p}^T M^{-1} \mathbf{p}\right]  \\
        \mathrm{Var}[U] &= \frac{D}{4 \alpha^2} \mathrm{Var}\left[\mathbf{p}^T \mathbf{p}\right] \\
        \mathrm{Var}[U] &= \frac{D}{4 \alpha^2} \mathrm{Var}\left[\sum_i^D{\mathbf{p}_i^2}\right] \\
        \mathrm{Var}[U] &= \frac{D^2}{4 \alpha^2} \alpha^4 \\
        \mathrm{Var}[U] &= \frac{D^2 \alpha^2}{4} \\
        \alpha = \frac{2}{D} \sqrt{\mathrm{Var}[U]}
    \end{aligned}
    \end{equation}
    Which gives the result for the metric
    \begin{equation}\label{eq:metric_adaption_appendix}
        M = \frac{2}{D} \sqrt{\mathrm{Var}[U]} \mathbb{1},
    \end{equation}

    \section{Code Implementation}\label{sec:code_implementation}

    \section{Auto Differentiation}\label{sec:autodiff}
    Auto differentiation is a modern approach to numerically calculating gradients.
    The essential idea is that a compiler can break down a program to its base operations and automatically apply the
    chain-rule repeatedly to calculate its gradient.
    Recent advancements in machine learning have greatly increased the demand for auto differentiation tools and now
    highly optimised frameworks exist to calculate gradients for any code base.

    One such example is

\end{document}