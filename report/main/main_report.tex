%%====================%%
%%  Part III Project  %%
%%====================%%

\documentclass[aps,prd,reprint,preprintnumbers,showpacs,floatfix,nofootinbib,superscript address]{revtex4-2}
\usepackage[utf8]{inputenc}
\usepackage{parskip}
\usepackage{amssymb}
\usepackage{stix}
\usepackage{hhline}
\usepackage{mathtools}
\usepackage[dvipsnames]{xcolor}
\usepackage{xspace}
\usepackage{multirow,tabularx}
\usepackage{siunitx}
\usepackage{graphicx}
\usepackage{xstring}
\usepackage{etoolbox}
\usepackage{notoccite}
\usepackage{natbib}
\usepackage{mathrsfs}
\usepackage{lineno}
\usepackage{tensor}
\usepackage{accents}
\usepackage{tikz}
\usepackage{listings}
\usepackage{algorithm}
\usepackage{algorithmic}
\usepackage{subcaption}
\allowdisplaybreaks

\parskip 1mm
\parindent 2mm

%! Author = borisdeletic
%! Date = 13/05/2023

% Preamble
\documentclass[11pt]{article}
\color



\definecolor{mygreen}{rgb}{0.0,0.6,0.0}
\definecolor{mygray}{rgb}{0.5,0.5,0.5}
\definecolor{mymauve}{rgb}{0.58,0,0.82}
\definecolor{background}{rgb}{1.0,1.0,1.0}
\definecolor{mywhite}{rgb}{0.0,0.0,0.0}

\lstset{
    backgroundcolor=\color{background},
    basicstyle=\footnotesize\color{mywhite},
    breakatwhitespace=false,
    breaklines=true,
    captionpos=b,
    commentstyle=\color{mygreen},
    deletekeywords={...},
    escapeinside={\%*}{*)},
    extendedchars=true,
    frame=single,
    keepspaces=true,
    keywordstyle=\color{blue},
    language=C++,
    morekeywords={*,...},
    numbersep=5pt,
    numberstyle=\tiny\color{mygray},
    rulecolor=\color{mywhite},
    showspaces=false,
    showstringspaces=false,
    showtabs=false,
    stringstyle=\color{mymauve},
    tabsize=2,
    title=\lstname
}
 %some personal macros

\usepackage{hyperref}
\hypersetup{%
     colorlinks = true,%
     linkcolor = Blue,%
     citecolor = Blue,%
     filecolor = Blue,%
     urlcolor = Blue% 
     }%
\usepackage[capitalize]{cleveref}
\usepackage{subfiles} %always load this last in preamble


\begin{document}

\title{Constrained Hamiltonian Monte Carlo with Nested Sampling for use in Lattice Field Theory}

\author{Candidate \#\#\#\#\#}
\affiliation{Astrophysics Group, Cavendish Laboratory, JJ Thomson Avenue, Cambridge CB3 0HE, UK}

\begin{abstract}
     Constrained Hamiltonian Monte Carlo is a novel algorithm for nested sampling in high dimensions by using gradients.
     We introduce a set of novel methods which modify traditional Hamiltonian Monte Carlo to be computationally viable for
     nested sampling.
     The algorithm is applied to numerically solve $\phi^4$-theory, a simple model in lattice field theory which
     exhibits a phase transition.
     We further demonstrate the resistance of our method to topological freezing and critical slowing down at criticality.
     In this work, we successfully measure observables on 512x512 sized lattices ($D=262,144$) using nested sampling.
\end{abstract}

\pacs{04.50.Kd, 04.60.-m, 04.20.Fy, or find your PACS numbers~\href{https://ufn.ru/en/pacs/}{here} and elsewhere.}

\maketitle

\subfile{1introduction}

\subfile{2nested_sampling}

\subfile{3hamiltonian_monte_carlo}

\subfile{4constrained_hmc}

\subfile{5parameter_adaption}

\subfile{6lattice_field_theory}

\subfile{7results}

\subfile{8performance}

\section{Conclusion}\label{sec:conclusion}
     We now have introduced our novel algorithm and implementation for Constrained Hamiltonian Monte Carlo for nested sampling.
     By drawing concepts from classical mechanics together with Bayesian inference we provide a powerful
     new set of tools which are very effectively applied to lattice field theory.

     Further work can be done in parallelising the implementation to fully take advantage of compute clusters.
     Using auto differentiation will also require further investigation to ensure no pathological behaviour arises for
     complex likelihoods.

     By exploiting gradients we believe it is possible to use this algorithm for a new class of
     Bayesian problems previously inaccessible to machines.


\bibliographystyle{apsrev4-1}
\bibliography{/Users/borisdeletic/CLionProjects/CHMC-Nested-Sampling/report/bibliography/references}

\subfile{appendix}

\end{document}
