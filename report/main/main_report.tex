%%====================%%
%%  Part III Project  %%
%%====================%%

\documentclass[aps,prd,reprint,preprintnumbers,showpacs,floatfix,nofootinbib,superscript address]{revtex4-2}
\usepackage[utf8]{inputenc}
\usepackage{parskip}
\usepackage{amssymb}
\usepackage{stix}
\usepackage{hhline}
\usepackage{mathtools}
\usepackage[dvipsnames]{xcolor}
\usepackage{xspace}
\usepackage{multirow,tabularx}
\usepackage{siunitx}
\usepackage{graphicx}
\usepackage{xstring}
\usepackage{etoolbox}
\usepackage{notoccite}
\usepackage{natbib}
\usepackage{mathrsfs}
\usepackage{lineno}
\usepackage{tensor}
\usepackage{accents}
\usepackage{tikz}
\usepackage{listings}
\usepackage{algorithm}
\usepackage{algorithmic}
\allowdisplaybreaks

\parskip 1mm
\parindent 2mm

%! Author = borisdeletic
%! Date = 13/05/2023

% Preamble
\documentclass[11pt]{article}
\color



\definecolor{mygreen}{rgb}{0.0,0.6,0.0}
\definecolor{mygray}{rgb}{0.5,0.5,0.5}
\definecolor{mymauve}{rgb}{0.58,0,0.82}
\definecolor{background}{rgb}{1.0,1.0,1.0}
\definecolor{mywhite}{rgb}{0.0,0.0,0.0}

\lstset{
    backgroundcolor=\color{background},
    basicstyle=\footnotesize\color{mywhite},
    breakatwhitespace=false,
    breaklines=true,
    captionpos=b,
    commentstyle=\color{mygreen},
    deletekeywords={...},
    escapeinside={\%*}{*)},
    extendedchars=true,
    frame=single,
    keepspaces=true,
    keywordstyle=\color{blue},
    language=C++,
    morekeywords={*,...},
    numbersep=5pt,
    numberstyle=\tiny\color{mygray},
    rulecolor=\color{mywhite},
    showspaces=false,
    showstringspaces=false,
    showtabs=false,
    stringstyle=\color{mymauve},
    tabsize=2,
    title=\lstname
}
 %some personal macros

\usepackage{hyperref}
\hypersetup{%
     colorlinks = true,%
     linkcolor = Blue,%
     citecolor = Blue,%
     filecolor = Blue,%
     urlcolor = Blue% 
     }%
\usepackage[capitalize]{cleveref}
\usepackage{subfiles} %always load this last in preamble


\begin{document}

\title{Constrained Hamiltonian Monte Carlo with Nested Sampling for use in Lattice Field Theory}

\author{Candidate \#\#\#\#\#}
\affiliation{Astrophysics Group, Cavendish Laboratory, JJ Thomson Avenue, Cambridge CB3 0HE, UK}
\affiliation{Kavli Institute for Cosmology, Madingley Road, Cambridge CB3 0HA, UK}
\affiliation{Institute of Astronomy, Madingley Road, Cambridge CB3 0HA, UK}
\affiliation{Department of Applied Mathematics and Theoretical Physics, Wilberforce Rd, Cambridge CB3 0WA, UK}

\begin{abstract}
	The abstract should be very short, just a single paragraph. You should convince the reader that the work is important and interesting. Avoid the use of \LaTeX{} in the abstract, which can cause problems with SEO later on. If absolutely necessary however, you can use some inline maths such as $E=mc^2$.
\end{abstract}

\pacs{04.50.Kd, 04.60.-m, 04.20.Fy, or find your PACS numbers~\href{https://ufn.ru/en/pacs/}{here} and elsewhere.}

\maketitle

%\begin{figure}[t!]
%  \center
%  \includegraphics[width=\linewidth]{PartIIIProjectTemplateFigure.pdf}
%	\caption{\label{PartIIIProjectTemplateFigure}
%	When captioning a figure, you should assume that the reader will have read the title and abstract of your paper and \textit{nothing else}. Therefore, the caption should be as self-contained as possible. Avoid acronyms, but feel free to use inline equations and even repeat content from the main body of the text. Most plots these days are produced via the Matplotlib library in Python. The plot above was made this way, but is not necessarily a good example of Matplotlib styling (which goes beyond the scope of this template). For schematic diagrams, flow-charts and suchlike, you should use \texttt{TikZ}. Never use the Wolfram language to produce plots.
%	}
%\end{figure}

\subfile{introduction}

\subfile{constrained_hmc}

\subfile{parameter_adaption}

\subfile{lattice_field_theory}


\bibliographystyle{apsrev4-1}
\bibliography{/Users/borisdeletic/CLionProjects/CHMC-Nested-Sampling/report/bibliography/references}

\subfile{appendix}

\end{document}
