%%====================%%
%%  Part III Project  %%
%%====================%%

\documentclass[aps,prd,reprint,preprintnumbers,showpacs,floatfix,nofootinbib,superscript address]{revtex4-2}
\usepackage[utf8]{inputenc}
\usepackage{parskip}
\usepackage{amssymb}
\usepackage{stix}
\usepackage{hhline}
\usepackage{mathtools}
\usepackage[dvipsnames]{xcolor}
\usepackage{xspace}
\usepackage{multirow,tabularx}
\usepackage{siunitx}
\usepackage{graphicx}
\usepackage{xstring}
\usepackage{etoolbox}
\usepackage{notoccite}
\usepackage{natbib}
\usepackage{mathrsfs}
\usepackage{lineno}
\usepackage{tensor}
\usepackage{accents}
\usepackage{tikz}
\usepackage{listings}
\usepackage{algorithm}
\usepackage{algorithmic}
\usepackage{caption}
\usepackage{ragged2e}  % for justification
\allowdisplaybreaks

\parskip 1mm
\parindent 2mm

%! Author = borisdeletic
%! Date = 13/05/2023

% Preamble
\documentclass[11pt]{article}
\color



\definecolor{mygreen}{rgb}{0.0,0.6,0.0}
\definecolor{mygray}{rgb}{0.5,0.5,0.5}
\definecolor{mymauve}{rgb}{0.58,0,0.82}
\definecolor{background}{rgb}{1.0,1.0,1.0}
\definecolor{mywhite}{rgb}{0.0,0.0,0.0}

\lstset{
    backgroundcolor=\color{background},
    basicstyle=\footnotesize\color{mywhite},
    breakatwhitespace=false,
    breaklines=true,
    captionpos=b,
    commentstyle=\color{mygreen},
    deletekeywords={...},
    escapeinside={\%*}{*)},
    extendedchars=true,
    frame=single,
    keepspaces=true,
    keywordstyle=\color{blue},
    language=C++,
    morekeywords={*,...},
    numbersep=5pt,
    numberstyle=\tiny\color{mygray},
    rulecolor=\color{mywhite},
    showspaces=false,
    showstringspaces=false,
    showtabs=false,
    stringstyle=\color{mymauve},
    tabsize=2,
    title=\lstname
}
 %some personal macros

\captionsetup{
     singlelinecheck=false,
     format=plain,
     justification = Justified
}

\usepackage{hyperref}
\hypersetup{%
     colorlinks = true,%
     linkcolor = Blue,%
     citecolor = Blue,%
     filecolor = Blue,%
     urlcolor = Blue% 
     }%
\usepackage[capitalize]{cleveref}
\usepackage{subfiles} %always load this last in preamble


\begin{document}

\title{Constrained Hamiltonian Monte Carlo with Nested Sampling for use in Lattice Field Theory}

\author{Candidate 8256T}
\affiliation{Astrophysics Group, Cavendish Laboratory, JJ Thomson Avenue, Cambridge CB3 0HE, UK}
\affiliation{Kavli Institute for Cosmology, Madingley Road, Cambridge CB3 0HA, UK}

\begin{abstract}
     Constrained Hamiltonian Monte Carlo is a novel algorithm for high dimensional nested sampling using gradients.
     We introduce a set of new methods which are used to demonstrate the first
     practical implementation incorporating gradients into nested sampling.
     Current state-of-the-art implementations of nested sampling become very limited for dimensions greater
     than $D > 100$.
     We show how our algorithm allows exploration of problems with dimension significantly higher, up to $D \sim 500,000$.

     We also use the algorithm in a novel application of nested sampling to $\phi^4$ lattice field theory,
     a simple model which exhibits a phase transition.
     We successfully measure observables on $128 \times 128$ sized lattices ($D=16,384$), exploring the behaviour at
     criticality.
     We further demonstrate the resistance of our method to topological freezing and critical slowing down.
\end{abstract}

\maketitle

\subfile{1introduction.tex}

\subfile{2bayesian_inference.tex}

\subfile{3nested_sampling.tex}

\subfile{4hamiltonian_monte_carlo.tex}

\subfile{5constrained_hmc.tex}

\subfile{6parameter_adaption.tex}

\subfile{7lattice_field_theory.tex}

\subfile{8results.tex}

\subfile{9performance.tex}

\subfile{10conclusion.tex}

\bibliographystyle{apsrev4-1}
\bibliography{/Users/borisdeletic/CLionProjects/CHMC-Nested-Sampling/report/bibliography/references}

\subfile{appendix}

\end{document}
