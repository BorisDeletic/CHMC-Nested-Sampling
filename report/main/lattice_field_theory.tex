%! Author = borisdeletic
%! Date = 07/05/2023

% Preamble
\documentclass[11pt]{article}

% Document
\begin{document}

\section{Scalar $\phi^4$ Theory on a Lattice}\label{sec:phi^4_theory}
    We work with a real, scalar $\phi^4$-theory in two-dimensions, discretized to a Euclidean square lattice of length $N$.
    The dimensionless action for the theory is given by
    \begin{equation}\label{eq:phi4_action}
    \begin{aligned}
        S = \,\sum\limits_{x \in \Lambda} \Biggl[-2\kappa \sum\limits_{\mu=1}^d & \phi(x) \phi(x+\hat{\mu}) \\
        &+\lambda \phi(x)^4 + (1 - 2\lambda) \phi(x)^2 \Biggr],
    \end{aligned}
    \end{equation}
    where $\Lambda$ is the set of lattice points, $\hat{\mu}$ is the unit vector in $\mu$ direction, and $\phi(x)$ is
    the field value at lattice site $x$~\cite{maas2020lattice}. $\kappa$ is the kinetic coupling for neighbour interactions and $\lambda$ is the
    coupling strength of the interaction.

    $\phi^4$ theory lives in the same universality class as the 2D Ising model, as we recover the Ising model
    in the limit $\lambda \rightarrow \infty$.
    As such, the model exhibits a second-order phase transition in the $\kappa, \lambda$ parameters, associated
    with the breaking of $\mathcal{Z}_2$ symmetry.
    The order parameter which undergoes the phase transition is the mean magnetization, defined as the expectation of
    absolute field value
    \begin{equation}\label{eq:magnetization}
        \langle M \rangle = \frac{1}{N^2} \bigl< \sum_{x \in \Lambda} |\phi(x)| \bigr>.
    \end{equation}

    The two point correlation function $\langle \phi(x_1) \phi(x_2) \rangle$ represents a physical observable of the
    quantum field theory, which are used to measure parameters of the system such as the correlation length,
    which diverges at the critical point.

    Beginning with the Wick-rotated path integral~\cite{maas2020lattice}
    \begin{equation}\label{eq:path_integral}
        \mathcal{Z} = \int {\mathcal{D}\phi e^{-S[\phi]}},
    \end{equation}
    we note that the factor $e^{-S[\phi]}$ can be interpreted as a probability weighting for a given field configuration.
    Therefore, defining the log likelihood function as $\log{\mathcal{L}} = -S[\phi]$, we reframe the problem in
    Bayesian inference terms, allowing us to apply the tools we have developed.

    \subsection{Nested Sampling for $\phi^4$ theory}\label{subsec:nested-sampling-phi4}
    The model of $\phi^4$-theory is a good candidate to test CHMC for nested sampling for a number of reasons.
    Below the critical point, the loglikelihood is a bimodal distribution allowing us to investigate the clustering
    behaviour of the algorithm.
    The likelihood gradient is also analytic
    \begin{equation}\label{eq:grad_likelihood}
    \begin{aligned}
        \frac{\partial \log{\mathcal{L}}} {\partial \phi(x)} = -2\kappa \sum\limits_{\mu} &\phi(x+\hat{\mu}) \\
        &+ 4\lambda \phi(x)^3 + 2(1-2\lambda)\phi(x).
    \end{aligned}
    \end{equation}
    Furthermore, as we approach the critical point, this model exhibits \emph{critical slowing down}, allowing us to
    test the resilience of the algorithm to this phenomenon.



\end{document}