%! Author = borisdeletic
%! Date = 03/05/2023

% Preamble
\documentclass[11pt]{article}

% Packages
\usepackage{amsmath}

% Document
\begin{document}

    \section{Introduction}\label{Introduction}
    Since the inception of Lattice Field Theory (LFT), numerical Monte-Carlo methods have been used with great success to
    study complex systems.
    The dominant algorithm used for simulations in LFT over the past decades has been Hamiltonian Monte Carlo
    (HMC)~\cite{HMC_Duane}, a Markov-Chain Monte-Carlo (MCMC) method used for \emph{parameter estimation}.
    HMC has since found wide applications throughout a number of different fields in statistics.

    To recover the physical behaviour of a lattice system, the continuum limit is taken as the system approaches its
    critical point.
    At the critical point, MCMC methods suffer from \emph{critical slowing down}~\cite{CriticalSlowingWOLFF} and
    \emph{topological freezing}~\cite{Hasenbusch_2018}, where all samples proposed by the algorithm are highly
    autocorrelated.
    Critical slowing down is a significant problem in LFT and Lattice QCD, with large efforts devoted to combating
    it's effects~\cite{Pawlowski_2020,Jansen_MLMC_2020,Albergo_Flow_LFT_2019}.

    A different approach to parameter estimation is with Bayesian inference.
    A contemporary method for Bayesian inference which has found many applications in astrophysics and cosmology
    is with nested sampling~\cite{Skilling2006,Handley_polychord}.
    Nested sampling offers several advantages to classic MCMC algorithms, including simultaneous calculation of model
    \emph{evidence}, as well as dealing effectively with multimodal distributions.
    Current implementations of nested sampling scale poorly with number of dimensions~\cite{Feroz_2009}, which quickly
    renders them ineffective for solving problems in LFT.

    We propose a novel algorithm, based off combining nested sampling with a modified constrained HMC, which aims
    to effectively sample in very high-dimensional parameter space, while being resistant to multimodal distributions
    and topological freezing.

    Describe structure of report


\end{document}